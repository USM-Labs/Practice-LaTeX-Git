\documentclass[a4paper,12pt]{report}

\usepackage[russian]{config}

% Переменные
\graphicspath{ {./images/} }
\newcommand{\uniGroupName}{I2402}

\newcommand{\authorNameRu}{Сплавский Максим Юрьевич}
\newcommand{\thesisTitleRu}{Изучение \LaTeX{} и Git}
\newcommand{\identificatorulCursuluiRu}{0613.4 Информатика}
\newcommand{\conducatorNameRu}{Курманский Антон}

\renewcommand{\year}{2025}
\newcommand{\github}{\url{https://github.com/USER/REPO}}
\newcommand{\pdfLink}{\url{https://github.com/USER/REPO/FILE.pdf}}
\newcommand{\outputDate}{\today}

\begin{document}

\begin{titlepage}
  \vspace*{\fill}
  \begin{center}
    \vspace*{1cm}

    \large
    \uppercase{\textbf{МОЛДАВСКИЙ ГОСУДАРСТВЕННЫЙ УНИВЕРСИТЕТ  \\}}

    \normalsize
    \uppercase{\textbf{ФАКУЛЬТЕТ МАТЕМАТИКИ И ИНФОРМАТИКИ}}
    \vspace{0.1cm}

    \normalsize
    \uppercase{\textbf{ДЕПАРТАМЕНТ ИНФОРМАТИКИ}}
    \vspace{3.0cm}

    \large
    \textbf{\correctName{\authorNameRu}}
    \vspace{1.5cm}

    \Large
    \textbf{\MakeUppercase{\thesisTitleRu}}
    \vspace{0.75cm}

    \large
    \textbf{\MakeUppercase{\identificatorulCursuluiRu}}
    \vspace{0.5cm}

    \normalsize
    \textbf{Практика}

  \end{center}
  \vfill

  \normalsize

  \begin{flushleft}
    \newcommand{\signatureField}{\textUnderLine{4cm}{(подпись)}}

    \begin{tabular}{p{4.5cm} p{4cm} p{8cm}}
      Директор департамента: & \signatureField{} & КАПЧЕЛЯ Титу, \\
                             &                   & доктор физико-математических наук, \\
                             &                   & преподаватель университета \\
      Научный руководитель:  & \signatureField{} & \correctName{\conducatorNameRu}, \\
                             &                   & ассистент университета \\
      Автор:                 & \signatureField{} & \correctName{\authorNameRu}, \\
                             &                   & студент группы \uniGroupName{} \\
    \end{tabular}

  \end{flushleft}

  \vspace{1cm}

  \begin{center}
    \textbf{КИШИНЕВ -- \year{} Г.}
  \end{center}

\end{titlepage}
 % Титульный лист

% -------------------------------------

\clearpage
\tableofcontents % Содержание

% -------------------------------------

\clearpage
\unnumberedChapter{Аннотация}

\textbf{Практическая работа на тему ``\thesisTitleRu{}'', студента \authorNameRu{}, группа \uniGroupName{}.}

\textbf{Структура практической работы.}
Практическая работа состоит из: Введения, трёх глав, Заключения с рекомендациями, библиографии и приложений. В работе представлены теоретические сведения, практические задания и иллюстрации, отражающие процесс изучения систем \LaTeX{} и Git. Основной текст включает подробные инструкции и скриншоты, демонстрирующие выполнение заданий на платформе Learn Git Branching \cite{learngitbranching}.

\textbf{Актуальность.}
Системы управления версиями и профессиональной вёрстки документов играют важную роль в современной IT-индустрии. Git широко используется в командной разработке программного обеспечения \cite{git_official_doc, progit_book}, а \LaTeX{} — в научной и инженерной документации \cite{latex_official_doc, latex_project_site, overleaf_intro}. Их знание необходимо для любого начинающего специалиста в сфере ИТ.

\textbf{Цель и задачи исследования.}
Целью данной работы является освоение основ использования \LaTeX{} и Git, а также практическое применение полученных знаний на платформе Learn Git Branching.

Для достижения цели были поставлены следующие задачи:
\begin{itemize}
  \item Изучить синтаксис и особенности вёрстки с использованием \LaTeX{} \cite{latex_official_doc, latex_project_site};
  \item Освоить ключевые команды Git и принципы работы с системой контроля версий \cite{progit_book};
  \item Пройти практические задания Learn Git Branching по ветвлению, слиянию и удалённым репозиториям \cite{learngitbranching};
  \item Подготовить отчётный документ в \LaTeX{}, используя современные пакеты, включая подсветку кода с помощью \texttt{minted} \cite{latex_minted}.
\end{itemize}

\textbf{Ожидаемые и полученные результаты.}
Ожидалось овладение базовыми навыками вёрстки и контроля версий. В результате был оформлен структурированный документ в \LaTeX{}, содержащий полную практическую часть по Git и Learn Git Branching, включая графические иллюстрации и разъяснения к каждому этапу.

\textbf{Важные решённые проблемы.}
В процессе работы были решены следующие задачи:
\begin{itemize}
  \item организация многоуровневой структуры документа в \LaTeX{};
  \item изучение моделей ветвления и командной работы в Git;
  \item визуализация выполнения практических заданий;
  \item интеграция технического и визуального материала в единую отчётную работу.
\end{itemize}

\textbf{Практическая ценность.}
Результаты работы могут быть использованы как руководство для студентов, начинающих осваивать \LaTeX{} и Git, а также как база для подготовки отчётной и научной документации в технических вузах.

Весь исходный код проекта доступен на GitHub по следующей ссылке: \github.

% -------------------------------------

\introChapter

\textbf{Актуальность и важность темы.}

Освоение систем \LaTeX{} и Git \cite{latex_official_doc, progit_book} является важным шагом в формировании технической грамотности современного IT-специалиста. Git применяется повсеместно в процессе разработки программного обеспечения, позволяя эффективно управлять изменениями и взаимодействовать в команде. \LaTeX{}, в свою очередь, обеспечивает профессиональную вёрстку документов, широко используемую в научной, инженерной и образовательной среде \cite{latex_project_site, overleaf_intro}.

Практическая направленность данной работы, подкреплённая выполнением заданий на интерактивной платформе Learn Git Branching \cite{learngitbranching}, делает изучение не только теоретически обоснованным, но и прикладным.

\textbf{Цель и задачи.}

Цель — получить практические навыки работы с \LaTeX{} и Git, а также закрепить знания через выполнение реальных задач, включая визуализацию ветвлений и взаимодействие с удалёнными репозиториями.

Задачи:
\begin{itemize}
  \item Изучить принципы и команды Git, в том числе ветвление, слияние и работу с удалёнными репозиториями;
  \item Освоить синтаксис \LaTeX{}, включая структуру документа, оформление кода, таблиц, рисунков;
  \item Подготовить отчёт с использованием современных пакетов (\texttt{minted}, \texttt{graphicx} и др.) \cite{latex_minted};
  \item Пройти все ключевые блоки платформы Learn Git Branching: Introduction, Ramping Up, Moving Work Around, Mixed Bag и Advanced Topics;
  \item Создать структурированный отчёт, включающий скриншоты выполнения заданий и объяснение решений.
\end{itemize}

\textbf{Методологическая и технологическая база.}

В работе использованы:
\begin{itemize}
  \item Платформа Learn Git Branching \cite{learngitbranching} для симуляции работы с Git;
  \item Компилятор XeLaTeX для создания PDF-документа;
  \item Пакеты \texttt{minted}, \texttt{graphicx}, \texttt{biblatex} для оформления кода, иллюстраций и библиографии;
  \item Git CLI и веб-интерфейс GitHub (в обучающей форме, без публикации проекта).
\end{itemize}

\textbf{Научная новизна / оригинальность.}

Работа сочетает сразу два технологических направления — вёрстку и контроль версий — и демонстрирует практическое применение обеих технологий в едином отчёте. Также показан прогрессивный подход к обучению Git через визуальное взаимодействие, что повышает понимание абстрактных концепций.

\textbf{Практическая ценность.}

Полученные материалы могут использоваться:
\begin{itemize}
  \item как пособие для студентов технических направлений;
  \item как пример оформления отчётов в \LaTeX{};
  \item как руководство по решению задач Git через Learn Git Branching.
\end{itemize}

\textbf{Краткое содержание работы.}

\begin{itemize}
  \item Первая глава знакомит с основами системы \LaTeX{}, её назначением, преимуществами и базовыми конструкциями;
  \item Вторая глава описывает Git, его возможности и применимость в командной разработке;
  \item Третья глава посвящена практическому выполнению задач на платформе Learn Git Branching с подробным разбором каждой секции.
\end{itemize}

% -------------------------------------

\chapter{Введение в систему вёрстки \LaTeX}\label{latex_introduction}

\section{Зачем нужен \LaTeX?}
\LaTeX — это система вёрстки документов, созданная для подготовки высококачественных текстов. Особенно эффективно используется в научной, технической и инженерной среде. Система позволяет:
\begin{itemize}[noitemsep]
    \item структурировать текст с помощью разделов, глав и подглав;
    \item оформлять сложные математические формулы;
    \item вставлять таблицы, рисунки и графики;
    \item управлять библиографией и ссылками;
    \item создавать презентации, слайды и даже резюме.
\end{itemize}

\section{История создания \LaTeX}
\LaTeX основан на системе \TeX, разработанной Дональдом Кнутом в 1978 году~\cite{latex_project_site}. В 1980-х Лесли Лэмпорт разработал \LaTeX как надстройку над \TeX, сделав её более удобной для повседневного использования учёными и инженерами. С тех пор \LaTeX стал стандартом для научных публикаций~\cite{latex_official_doc}.

\section{Преимущества и ключевые особенности}
\begin{itemize}[noitemsep]
    \item Полная автоматизация оформления (оглавления, списки рисунков, библиография).
    \item Идеальная типографика (переносы, выравнивание, интервалы).
    \item Гибкость в настройке внешнего вида.
    \item Возможность написания собственных макросов и пакетов.
    \item Совместимость с Git и другими системами контроля версий (текстовый формат).
\end{itemize}

\section{Основные команды \LaTeX}
\begin{itemize}[noitemsep]
    \item \verb|\documentclass{article}| — выбор типа документа (\verb|book|, \verb|report|, \verb|beamer| и др.).
    \item \verb|\usepackage{graphicx}| — подключение пакета для вставки изображений.
    \item \verb|\section{Название}|, \verb|\subsection{...}|, \verb|\subsubsection{...}| — структура документа.
    \item \verb|\textbf{жирный}|, \verb|\emph{курсив}|, \verb|\underline{подчёркнутый}| — оформление текста.
    \item \verb|\begin{itemize}...\end{itemize}| — маркированный список.
    \item \verb|\begin{enumerate}...\end{enumerate}| — нумерованный список.
    \item \verb|\begin{table}...\end{table}| — таблицы с подписями.
    \item \verb|\begin{figure}...\includegraphics{...}\end{figure}| — вставка изображений.
    \item \verb|\begin{equation} a^2 + b^2 = c^2 \end{equation}| — математическая формула.
    \item \verb|\label{...}|, \verb|\ref{...}|, \verb|\cite{...}| — система ссылок.
    \item \verb|\bibliography{...}| и \verb|\addbibresource{...}| — подключение библиографии.
\end{itemize}

\section{Дополнительные возможности}
\begin{itemize}[noitemsep]
    \item Работа с цветом: \verb|\usepackage{xcolor}|, затем \verb|\textcolor{red}{text}|.
    \item Создание графики внутри \LaTeX: \verb|\usepackage{tikz}|.
    \item Создание слайдов: класс \verb|beamer|.
    \item Код и подсветка синтаксиса: \verb|\usepackage{minted}| или \verb|listings|~\cite{latex_minted}.
    \item Многоязычность: \verb|\usepackage[main=russian,english]{babel}|.
    \item Индексы и оглавление: \verb|\tableofcontents|, \verb|\printindex|.
    \item Формулы в строке и на отдельной строке: \verb|$...$| и \verb|\[ ... \]| соответственно.
\end{itemize}

\section{Редакторы и компиляция}
Существует множество способов работы с \LaTeX. Наиболее популярные редакторы:
\begin{itemize}[noitemsep]
    \item Overleaf — онлайн-платформа с поддержкой совместной работы~\cite{overleaf_intro}.
    \item TeXstudio — классический офлайн-редактор.
    \item VS Code + LaTeX Workshop — модульная и гибкая среда.
\end{itemize}
Компиляция может осуществляться движками \verb|pdflatex|, \verb|xelatex|, \verb|lualatex| в зависимости от нужд (языки, шрифты и др.).

\section{Выводы}
\LaTeX — мощный инструмент, требующий начального обучения, но обеспечивающий выдающееся качество выходного документа. Он идеален для научных, технических и структурно сложных работ.

\section*{Источники}
\begin{itemize}[noitemsep]
    \item \cite{latex_project_site} — официальный сайт проекта LaTeX.
    \item \cite{latex_official_doc} — документация LaTeX2e.
    \item \cite{overleaf_intro} — платформа Overleaf.
    \item \cite{latex_minted} — информация о пакете minted для подсветки кода.
\end{itemize}

\chapterConclusionSection{latex_introduction}

В первой главе была рассмотрена система вёрстки \LaTeX{}, её структура, синтаксис и основные преимущества. Изучены базовые команды, правила форматирования текста, оформление списков, таблиц, формул и изображений. Также были протестированы возможности расширения через подключение внешних пакетов, таких как \texttt{graphicx} для работы с графикой и \texttt{minted} для подсветки кода \cite{latex_minted}.

Работа с \LaTeX{} требует определённой дисциплины и внимания к структуре документа, но взамен предоставляет высокое качество выходного материала и гибкость при масштабировании проектов. В результате глава обеспечила фундамент для дальнейшего использования \LaTeX{} при оформлении технической документации.

% -------------------------------------

\chapter{Введение в систему контроля версий Git}\label{git_introduction}

\section{Зачем нужен Git?}
Git — это распределённая система контроля версий, созданная для эффективного управления изменениями в проектах с большим количеством исходного кода. Он позволяет:
\begin{itemize}[noitemsep]
    \item отслеживать историю изменений;
    \item работать в команде без конфликтов;
    \item откатываться к предыдущим версиям кода;
    \item создавать альтернативные ветви разработки (branching);
    \item автоматически сливать изменения (merging).
\end{itemize}

\section{История создания Git}
Git был создан Линусом Торвальдсом в 2005 году для управления исходным кодом ядра Linux после конфликта с предыдущей системой контроля версий — BitKeeper~\cite{progit_book}. Требования:
\begin{itemize}[noitemsep]
    \item высокая скорость работы;
    \item надёжная защита от потерь данных;
    \item поддержка распределённой архитектуры;
    \item лёгкость в ветвлении и слиянии.
\end{itemize}

\section{Ключевые особенности Git}
\begin{itemize}[noitemsep]
    \item Каждый разработчик имеет полную копию репозитория (локально).
    \item Все изменения сохраняются в виде снапшотов.
    \item Ветвление и слияние — базовая часть рабочего процесса.
    \item Минимальная зависимость от центрального сервера.
\end{itemize}

Подробнее о философии Git — в книге Pro Git~\cite{progit_book}.

\section{Основные команды Git}
\begin{itemize}[noitemsep]
    \item \texttt{git init} — инициализация репозитория.
    \item \texttt{git clone} — копирование удалённого репозитория.
    \item \texttt{git status} — просмотр состояния файлов.
    \item \texttt{git add} — добавление файлов в индекс.
    \item \texttt{git commit} — фиксация изменений.
    \item \texttt{git branch} — работа с ветками.
    \item \texttt{git checkout} — переключение между ветками.
    \item \texttt{git merge} — слияние веток.
    \item \texttt{git pull}, \texttt{git push} — взаимодействие с удалёнными репозиториями.
\end{itemize}

Команды подробно описаны в официальной документации~\cite{git_official_doc}.

\section{Источники и полезные ссылки}
\begin{itemize}[noitemsep]
    \item \cite{git_official_doc} — официальная документация Git.
    \item \cite{learngitbranching} — платформа интерактивного обучения Learn Git Branching.
    \item \cite{progit_book} — Pro Git book.
\end{itemize}

\chapterConclusionSection{git_introduction}

Вторая глава была посвящена изучению системы контроля версий Git. Рассмотрены её архитектура, принципы работы и основные команды. Были объяснены понятия коммита, индексации, истории изменений, а также таких операций, как ветвление (branching), слияние (merging) и откат изменений (reset, revert) \cite{progit_book, git_official_doc}.

Git продемонстрировал себя как надёжный инструмент для индивидуальной и командной разработки. Его использование позволяет систематизировать процесс разработки, минимизировать риски потери данных и упростить интеграцию изменений. Глава заложила основу для выполнения практических задач на платформе Learn Git Branching.

% -------------------------------------

\chapter{Практика с Learn Git Branching}\label{git_branching_practice}

В качестве обучающей платформы используется интерактивный тренажёр Learn Git Branching~\cite{learngitbranching}.

\textbf{Важно:} последующая работа подразумевает что читатель уже ознакомился с источниками~\cite{git_official_doc, progit_book, learngitbranching}.

\section{Main — Базовые концепции Git}

\subsection{Introduction Sequence}
\subsubsection{Introduction to Git Commits}
\textbf{Цель:} познакомиться с понятием коммита в Git.

Коммит в Git \textendash{} это снимок состояния всех отслеживаемых файлов. Git старается хранить только изменения (дельты), чтобы минимизировать использование памяти и ускорить операции. Каждый коммит связан с предыдущим (родительским), что позволяет формировать полную историю проекта.

\textbf{Команда:}
\begin{minted}{bash}
git commit
\end{minted}

\refFigure{screenshot_LGB_1-1-1.png} представлен скриншот данного задания.
\imageWithCaption{screenshot_LGB_1-1-1.png}{}

После выполнения создаётся новый коммит, например \texttt{C2}, ссылающийся на \texttt{C1} как на родителя.

\subsubsection{Branching in Git}
\textbf{Цель:} научиться создавать и переключаться между ветками.

Ветки в Git \textendash{} это указатели на определённые коммиты. Они практически не занимают места, поэтому создавать много веток \textendash{} это нормально и даже рекомендуется.

\textbf{Команды:}
\begin{minted}{bash}
git branch newImage   # создание новой ветки
git checkout newImage # переключение на неё
git commit            # коммит будет в ветке newImage
\end{minted}

Альтернатива:
\begin{minted}{bash}
git checkout -b bugFix
\end{minted}

\refFigure{screenshot_LGB_1-1-2.png} представлен скриншот данного задания.
\imageWithCaption{screenshot_LGB_1-1-2.png}{}

\textbf{Важно:} коммиты записываются только в активную ветку (ту, где стоит символ *).

\subsubsection{Merging in Git}
\textbf{Цель:} объединить работу из разных веток в одну.

Merge создаёт специальный коммит с двумя родителями, включающий изменения из обеих веток. Это особенно полезно для совместной работы.

\textbf{Команды:}
\begin{minted}{bash}
git merge bugFix       # слияние ветки bugFix в текущую (main)
git checkout bugFix    # переход на другую ветку
git merge main         # слияние main в bugFix
\end{minted}

\refFigure{screenshot_LGB_1-1-3.png} представлен скриншот данного задания.
\imageWithCaption{screenshot_LGB_1-1-3.png}{}

После слияния все ветки включают весь набор коммитов проекта.

\subsubsection{Rebase Introduction}
\textbf{Цель:} познакомиться с альтернативным способом объединения веток.

Rebase \textendash{} это перемещение коммитов из одной ветки на основание другой. Это делает историю проекта линейной и упрощает её анализ. Однако следует использовать осторожно, особенно при работе с общими ветками.

\textbf{Команды:}
\begin{minted}{bash}
git rebase main        # переместить коммиты bugFix поверх main
git checkout main
git rebase bugFix      # переместить main после bugFix
\end{minted}

\refFigure{screenshot_LGB_1-1-4.png} представлен скриншот данного задания.
\imageWithCaption{screenshot_LGB_1-1-4.png}{}

\textbf{Важно:} Rebase изменяет историю коммитов и создаёт новые идентификаторы для "перемещённых" коммитов.

Дополнительную информацию о командах и концепциях Git можно найти в документации и книге Pro Git.

\subsection{Ramping Up}

\subsubsection{Detach yo' HEAD}
\textbf{Цель:} понять, как работает HEAD и что означает его "отсоединение".

В Git переменная \texttt{HEAD} указывает на текущий коммит или ветку, на которой вы находитесь. Обычно она указывает на имя ветки (например, \texttt{main}), но можно напрямую привязать HEAD к коммиту.

Это называется "отсоединённым HEAD" (detached HEAD), и в этом режиме коммиты не записываются ни в какую ветку, пока не будет выполнено явно указание.

\textbf{Команды:}
\begin{minted}{bash}
git checkout C1        # переход на конкретный коммит
\end{minted}

\textbf{Результат:} HEAD указывает напрямую на коммит \texttt{C1}, а не на ветку.

\textbf{Практика:} Проверьте команды:
\begin{minted}{bash}
git checkout main
git commit
\end{minted}

Затем отсоедините HEAD:
\begin{minted}{bash}
git checkout C2
\end{minted}

\refFigure{screenshot_LGB_1-2-1.png} представлен скриншот данного задания.
\imageWithCaption{screenshot_LGB_1-2-1.png}{}

\subsubsection{Relative Refs (\texttt{\^{}})}
\textbf{Цель:} освоить навигацию по коммитам с помощью относительных ссылок.

Вместо указания полного хеша коммита, Git позволяет перемещаться по дереву коммитов с помощью символов \texttt{\^{}} и \texttt{\~{}}:

- \texttt{main\^} — родитель \texttt{main};
- \texttt{main\^{}}\texttt{\^{}} — дедушка;
- \texttt{HEAD\^} — один шаг назад от текущего коммита.

\textbf{Команды:}
\begin{minted}{bash}
git checkout main^
git checkout HEAD^
git checkout HEAD^
\end{minted}

\refFigure{screenshot_LGB_1-2-2.png} представлен скриншот данного задания.
\imageWithCaption{screenshot_LGB_1-2-2.png}{}

\textbf{Результат:} Вы перемещаетесь назад по истории коммитов.

\subsubsection{Relative Refs (\texttt{\~})}
\textbf{Цель:} быстро перемещаться назад по истории на несколько шагов.

\texttt{HEAD\textasciitilde 3} эквивалентно \texttt{HEAD\^{}\^{}\^{}}, но значительно короче.

\textbf{Команды:}
\begin{minted}{bash}
git checkout HEAD~4
\end{minted}

Можно использовать это в связке с \texttt{-f} (force) для перемещения веток:
\begin{minted}{bash}
git branch -f main HEAD~3
\end{minted}

\refFigure{screenshot_LGB_1-2-3.png} представлен скриншот данного задания.
\imageWithCaption{screenshot_LGB_1-2-3.png}{Скиншот задания}

\textbf{Примечание:} В реальных условиях запрещено перемещать текущую ветку таким способом — сначала нужно переключиться на другую.

\subsubsection{Reversing Changes in Git}
\textbf{Цель:} научиться отменять изменения с помощью \texttt{reset} и \texttt{revert}.

\textbf{\texttt{git reset}} — откат ветки на предыдущий коммит (используется для локальной работы).

\begin{minted}{bash}
git reset HEAD~1
\end{minted}

\textbf{\texttt{git revert}} — создаёт новый коммит, отменяющий предыдущий (удобно для публичных веток).

\begin{minted}{bash}
git revert HEAD
\end{minted}

\refFigure{screenshot_LGB_1-2-4.png} представлен скриншот данного задания.
\imageWithCaption{screenshot_LGB_1-2-4.png}{Скиншот задания}

\textbf{Вывод:} Reset переписывает историю. Revert создаёт откат как отдельный коммит и безопасен для совместной работы.

\subsection{Moving Work Around}

\subsubsection{Cherry-pick Intro}
\textbf{Цель:} научиться переносить отдельные коммиты между ветками.

Команда \texttt{git cherry-pick} позволяет выбрать один (или несколько) коммитов и "применить" их в текущую ветку. Это удобно, когда нужно скопировать конкретные изменения, не сливая всю ветку целиком.

\textbf{Сценарий:} допустим, нужный коммит есть в ветке \texttt{feature}, но мы хотим перенести его в \texttt{main} без merge или rebase.

\textbf{Команда:}
\begin{minted}{bash}
git cherry-pick <hash-коммита>
\end{minted}

\textbf{Пример:}
\begin{minted}{bash}
git checkout main
git cherry-pick C4
\end{minted}

\refFigure{screenshot_LGB_1-3-1.png} представлен скриншот данного задания.
\imageWithCaption{screenshot_LGB_1-3-1.png}{}

\textbf{Результат:} Ветка \texttt{main} получит копию коммита \texttt{C4}.

\textbf{Важно:} коммит получает новый hash (так как история изменяется).

\subsubsection{Interactive Rebase Intro}
\textbf{Цель:} познакомиться с интерактивной перезаписью истории коммитов.

\texttt{git rebase -i} позволяет:
\begin{itemize}[noitemsep]
    \item изменить порядок коммитов;
    \item объединить коммиты (squash);
    \item удалить ненужные коммиты;
    \item изменить сообщения к коммитам.
\end{itemize}

\textbf{Команда:}
\begin{minted}{bash}
git rebase -i HEAD~3
\end{minted}

После запуска откроется список последних 3 коммитов. В интерактивном режиме можно выбрать действия: \texttt{pick}, \texttt{reword}, \texttt{edit}, \texttt{squash}, \texttt{drop}.

\textbf{Пример:} чтобы объединить два последних коммита:
\begin{minted}{text}
pick 1a2b3c Первый коммит
squash 4d5e6f Второй коммит
\end{minted}

\refFigure{screenshot_LGB_1-3-2.png} представлен скриншот данного задания.
\imageWithCaption{screenshot_LGB_1-3-2.png}{}

\textbf{Результат:} оба коммита объединятся в один.

\textbf{Важно:} интерактивный rebase лучше использовать только в локальных ветках, до публикации в удалённый репозиторий.

\subsection{A Mixed Bag}

\subsubsection{Grabbing Just 1 Commit}
\textbf{Цель:} изолированно скопировать один конкретный коммит из другой ветки.

Иногда необходимо взять только один коммит из другой ветки, не объединяя всё её содержимое. Это решается с помощью \texttt{git cherry-pick}.

\textbf{Команды:}
\begin{minted}{bash}
git checkout main
git cherry-pick <hash-коммита>
\end{minted}

\refFigure{screenshot_LGB_1-4-1.png} представлен скриншот данного задания.
\imageWithCaption{screenshot_LGB_1-4-1.png}{}

\textbf{Сценарий:} один коммит на ветке содержит багфикс — берём только его.

\subsubsection{Juggling Commits}
\textbf{Цель:} научиться перемещать ветки и коммиты с помощью \texttt{branch -f} и \texttt{checkout}.

Можно переназначать ветки на другие коммиты, используя ключ \texttt{-f} (force). Это мощный инструмент для манипуляций в истории.

\textbf{Команды:}
\begin{minted}{bash}
git branch -f main HEAD~2
git checkout C1
\end{minted}

\refFigure{screenshot_LGB_1-4-2.png} представлен скриншот данного задания.
\imageWithCaption{screenshot_LGB_1-4-2.png}{}

\textbf{Примечание:} подобные операции применимы только к локальным веткам.

\subsubsection{Juggling Commits \#2}
\textbf{Цель:} повторить и закрепить работу с переопределением веток.

Продолжение предыдущего задания. Здесь нужно грамотно совместить \texttt{checkout}, \texttt{commit} и \texttt{branch -f} для создания точной структуры коммитов.

\textbf{Команды:}
\begin{minted}{bash}
git checkout C0
git branch -f bugFix HEAD
git checkout bugFix
git commit
\end{minted}

\refFigure{screenshot_LGB_1-4-3.png} представлен скриншот данного задания.
\imageWithCaption{screenshot_LGB_1-4-3.png}{}

\textbf{Результат:} создаётся нужная ветка на базе старого коммита с новым изменением.

\subsubsection{Git Tags}
\textbf{Цель:} познакомиться с тегами в Git и их созданием.

Тег (tag) — это постоянная метка на конкретном коммите. Используется для пометки релизов, важных версий и контрольных точек.

\textbf{Команда:}
\begin{minted}{bash}
git tag v1 C1
\end{minted}

\refFigure{screenshot_LGB_1-4-4.png} представлен скриншот данного задания.
\imageWithCaption{screenshot_LGB_1-4-4.png}{}

\textbf{Примечание:} в отличие от ветки, тег не перемещается при новых коммитах.

\subsubsection{Git Describe}
\textbf{Цель:} научиться использовать \texttt{git describe} для понимания положения HEAD относительно тегов.

\texttt{git describe} показывает ближайший тег и расстояние (число коммитов) от него до текущего состояния ветки.

\textbf{Команда:}
\begin{minted}{bash}
git describe
\end{minted}

\textbf{Пример вывода:}
\begin{minted}{text}
v1-2-gabcdef
\end{minted}

\refFigure{screenshot_LGB_1-4-5.png} представлен скриншот данного задания.
\imageWithCaption{screenshot_LGB_1-4-5.png}{}

\texttt{v1} — ближайший тег, \texttt{2} — количество коммитов после тега, \texttt{gabcdef} — hash текущего коммита.

\subsection{Advanced Topics}

\subsubsection{Rebasing over 9000 times}
\textbf{Цель:} повторно закрепить использование \texttt{git rebase} и научиться выстраивать линейную историю при множественных ветках.

При сложной структуре ветвлений и параллельных разработках часто необходимо "переносить" изменения с ветки на ветку, чтобы сохранить читаемость истории и избежать конфликтов при слиянии.

\textbf{Команды:}
\begin{minted}{bash}
git rebase main
\end{minted}

\refFigure{screenshot_LGB_1-5-1.png} представлен скриншот данного задания.
\imageWithCaption{screenshot_LGB_1-5-1.png}{}

\textbf{Примечание:} важно выполнять rebase с осознанием того, что вы меняете историю коммитов. Не используйте на общедоступных ветках.

\subsubsection{Multiple Parents}
\textbf{Цель:} изучить коммиты с несколькими родителями (merge-коммиты).

Git позволяет объединять несколько веток, создавая коммит с двумя и более родителями. Это удобно, но влечёт за собой визуальную сложность истории.

\textbf{Команда:}
\begin{minted}{bash}
git merge feature1
\end{minted}

\refFigure{screenshot_LGB_1-5-2.png} представлен скриншот данного задания.
\imageWithCaption{screenshot_LGB_1-5-2.png}{}

\textbf{Пример:} если обе ветки имеют уникальные коммиты, результатом будет коммит с двумя родителями — \texttt{C4 = merge(C2, C3)}.

\subsubsection{Branch Spaghetti}
\textbf{Цель:} визуализировать и понять последствия хаотичной работы с ветками.

Когда разработчики ветвятся без правил и контроля, история превращается в "спагетти" — множество пересечений, нестабильность и путаница.

Цель уровня — выявить, как избежать этой ситуации и как можно аккуратно провести слияния или rebase для восстановления структуры.

\textbf{Рекомендации:}
\begin{itemize}[noitemsep]
  \item Используйте feature-ветки.
  \item Делайте squash перед merge.
  \item Поддерживайте линейную историю на основной ветке.
\end{itemize}

\refFigure{screenshot_LGB_1-5-3.png} представлен скриншот данного задания.
\imageWithCaption{screenshot_LGB_1-5-3.png}{}

\section{Remote — Работа с удалёнными репозиториями}

\subsection{Push \& Pull — Git Remotes}

\subsubsection{Clone Intro}
\textbf{Цель:} понять суть удалённого репозитория и создать его копию локально.

Команда \texttt{git clone} создаёт локальную копию удалённого репозитория. В рамках LearnGitBranching она используется для создания виртуального удалённого репозитория и отображения его структуры.

\textbf{Команда:}
\begin{minted}{bash}
git clone
\end{minted}

\refFigure{screenshot_LGB_2-1-1.png} представлен скриншот данного задания.
\imageWithCaption{screenshot_LGB_2-1-1.png}{}

\textbf{Результат:} появляется удалённый репозиторий (remote), связанный с текущим.

\subsubsection{Remote Branches}
\textbf{Цель:} изучить поведение удалённых веток.

Удалённые ветки (например, \texttt{o/main}) отражают состояние удалённого репозитория. Они обновляются после fetch/pull и не могут быть изменены напрямую.

\textbf{Пример поведения:}
\begin{minted}{bash}
git checkout o/main
git commit
\end{minted}

\refFigure{screenshot_LGB_2-1-2.png} представлен скриншот данного задания.
\imageWithCaption{screenshot_LGB_2-1-2.png}{}

\textbf{Результат:} создаётся коммит, но он не привязывается к ветке \texttt{o/main}, так как HEAD отсоединён.

\subsubsection{Git Fetchin'}
\textbf{Цель:} получить изменения с удалённого репозитория без их применения.

\texttt{git fetch} загружает изменения и обновляет удалённые ветки, но не изменяет локальные.

\textbf{Команда:}
\begin{minted}{bash}
git fetch
\end{minted}

\textbf{Поведение:}
\begin{itemize}[noitemsep]
    \item загружает недостающие коммиты;
    \item обновляет \texttt{o/main} (и др.);
    \item не влияет на ваш рабочий каталог.
\end{itemize}

\refFigure{screenshot_LGB_2-1-3.png} представлен скриншот данного задания.
\imageWithCaption{screenshot_LGB_2-1-3.png}{}

\subsubsection{Git Pullin'}
\textbf{Цель:} объединить fetch и merge в одну команду.

\texttt{git pull} = \texttt{git fetch} + \texttt{git merge}. Используется для актуализации локальной ветки с удалённой.

\textbf{Команды:}
\begin{minted}{bash}
git pull
git fetch; git merge o/main
\end{minted}

\refFigure{screenshot_LGB_2-1-4.png} представлен скриншот данного задания.
\imageWithCaption{screenshot_LGB_2-1-4.png}{}

\textbf{Результат:} локальная ветка будет дополнена новыми коммитами из \texttt{o/main}.

\subsubsection{Faking Teamwork}
\textbf{Цель:} смоделировать параллельную работу нескольких разработчиков.

В этом упражнении мы тренируемся в синхронизации изменений между локальной и удалённой ветками. Один разработчик делает коммит локально, другой — на удалённом.

\textbf{Команды:}
\begin{minted}{bash}
git commit             # локальный коммит
git fetch              # подтянуть удалённый
\end{minted}

\refFigure{screenshot_LGB_2-1-5.png} представлен скриншот данного задания.
\imageWithCaption{screenshot_LGB_2-1-5.png}{}

\textbf{Задача:} объединить обе версии через merge или rebase.

\subsubsection{Git Pushin'}
\textbf{Цель:} передать изменения в удалённый репозиторий.

\texttt{git push} отправляет ваши локальные коммиты в удалённую ветку. Только если история не расходится.

\textbf{Команда:}
\begin{minted}{bash}
git push
\end{minted}

\refFigure{screenshot_LGB_2-1-6.png} представлен скриншот данного задания.
\imageWithCaption{screenshot_LGB_2-1-6.png}{}

\textbf{Важно:} если история различается, push будет отклонён.

\subsubsection{Diverged History}
\textbf{Цель:} разрешить конфликты при расхождении истории.

Когда локальная и удалённая ветки имеют разные изменения, Git требует ручного вмешательства. Нужно выполнить:

\textbf{Команды:}
\begin{minted}{bash}
git pull --rebase      # или git fetch + rebase
\end{minted}

\textbf{После:} можно безопасно выполнить \texttt{git push}.

\refFigure{screenshot_LGB_2-1-7.png} представлен скриншот данного задания.
\imageWithCaption{screenshot_LGB_2-1-7.png}{}

\subsubsection{Locked Main}
\textbf{Цель:} изучить ситуацию, когда push запрещён без обновления локальной истории.

Некоторые удалённые репозитории (например, GitHub) запрещают push, если локальная ветка не включает последние изменения.

\textbf{Решение:}
\begin{minted}{bash}
git pull --rebase
\end{minted}

\refFigure{screenshot_LGB_2-1-8.png} представлен скриншот данного задания.
\imageWithCaption{screenshot_LGB_2-1-8.png}{}

\textbf{Результат:} локальная ветка обновляется и вы можете \texttt{push} без ошибок.

\subsection{To Origin and Beyond — Advanced Git Remotes}

\subsubsection{Push Main!}
\textbf{Цель:} закрепить основную команду для публикации ветки.

Задание фокусируется на использовании \texttt{git push} для основной ветки (main).

\textbf{Команда:}
\begin{minted}{bash}
git push origin main
\end{minted}

\refFigure{screenshot_LGB_2-2-1.png} представлен скриншот данного задания.
\imageWithCaption{screenshot_LGB_2-2-1.png}{}

\textbf{Примечание:} при явном указании ветки повышается контроль.

\subsubsection{Merging with remotes}
\textbf{Цель:} объединить удалённые изменения с локальными.

В случае, когда на удалённой стороне появились коммиты, которых нет у вас, нужно сначала их подтянуть:
\begin{minted}{bash}
git pull --rebase
\end{minted}

Затем можно безопасно делать:
\begin{minted}{bash}
git push
\end{minted}

\refFigure{screenshot_LGB_2-2-2.png} представлен скриншот данного задания.
\imageWithCaption{screenshot_LGB_2-2-2.png}{}

\subsubsection{Remote Tracking}
\textbf{Цель:} понять, как ветки отслеживают друг друга.

При клонировании создаются локальные ветки, "отслеживающие" удалённые:
\begin{minted}{bash}
git branch -vv
\end{minted}

\refFigure{screenshot_LGB_2-2-3.png} представлен скриншот данного задания.
\imageWithCaption{screenshot_LGB_2-2-3.png}{}

\textbf{Вывод:} вы увидите, какие ветки связаны с удалёнными и на сколько коммитов они отстают или опережают.

\subsubsection{Git push arguments}
\textbf{Цель:} научиться задавать явно, что и куда отправляется.

Формат:
\begin{minted}{bash}
git push <remote> <source>:<destination>
\end{minted}

\textbf{Пример:}
\begin{minted}{bash}
git push origin bugFix:main
\end{minted}

Это отправит ветку \texttt{bugFix} в удалённую \texttt{main}.

\refFigure{screenshot_LGB_2-2-4.png} представлен скриншот данного задания.
\imageWithCaption{screenshot_LGB_2-2-4.png}{}

\subsubsection{Git push arguments -- Expanded!}
\textbf{Цель:} управлять историей удалённого репозитория.

Удаление ветки с сервера:
\begin{minted}{bash}
git push origin :feature1
\end{minted}

\refFigure{screenshot_LGB_2-2-5.png} представлен скриншот данного задания.
\imageWithCaption{screenshot_LGB_2-2-5.png}{}

\textbf{Результат:} удалённая ветка \texttt{feature1} будет удалена.

\subsubsection{Fetch arguments}
\textbf{Цель:} частично обновлять удалённые данные.

Можно получать не все изменения, а только конкретные ветки:
\begin{minted}{bash}
git fetch origin bugFix
\end{minted}

\refFigure{screenshot_LGB_2-2-6.png} представлен скриншот данного задания.
\imageWithCaption{screenshot_LGB_2-2-6.png}{}

\subsubsection{Source of nothing}
\textbf{Цель:} проанализировать ситуацию, когда вы клонируете пустой репозиторий.

После \texttt{git clone} не будет ни одного коммита или ветки. Вы должны создать начальный коммит вручную:
\begin{minted}{bash}
git commit
\end{minted}

\refFigure{screenshot_LGB_2-2-7.png} представлен скриншот данного задания.
\imageWithCaption{screenshot_LGB_2-2-7.png}{}

\subsubsection{Pull arguments}
\textbf{Цель:} точно управлять направлением слияния.

Можно явно указать, откуда и куда тянуть изменения:
\begin{minted}{bash}
git pull origin main
\end{minted}

Это тянет ветку \texttt{main} с \texttt{origin} в текущую локальную ветку.

\refFigure{screenshot_LGB_2-2-8.png} представлен скриншот данного задания.
\imageWithCaption{screenshot_LGB_2-2-8.png}{}

\textbf{Заключение:} продвинутое взаимодействие с удалёнными репозиториями требует понимания того, что и куда передаётся. Явные аргументы повышают контроль и снижают ошибки.

\chapterConclusionSection{git_branching_practice}

В третьей главе была реализована практическая часть изучения Git на платформе Learn Git Branching \cite{learngitbranching}. Пользователь прошёл последовательно все основные модули: от базового ветвления и перемещения коммитов до работы с удалёнными репозиториями. Каждое задание было проанализировано и снабжено скриншотом, отражающим прогресс и логику решений.

Практика показала, что визуальный подход к обучению Git значительно ускоряет понимание концепций и развивает мышление, ориентированное на управление историями изменений. Итогом главы стало закрепление всех ключевых команд и принципов Git в интерактивной форме, что подтверждает успешное освоение материала и готовность применять знания в реальных проектах.

% -------------------------------------

\chapter{Git Immersion — Пошаговое погружение в Git}\label{git_immersive_practice}

Git Immersion — это обучающий интерактивный курс, созданный для пошагового изучения основных команд и концепций Git. Он построен в виде последовательных лабораторных заданий (Labs), каждая из которых направлена на закрепление определённого навыка: от базовой настройки и создания коммитов до анализа истории и работы с ветками. Практика ориентирована на начинающих пользователей и предполагает выполнение команд в командной строке с пояснениями результатов.

\section{Lab 1: Setup}
\textbf{Цель:} подготовить Git и Ruby для последующей работы.

\textbf{Настройка имени и электронной почты}

Если вы используете Git впервые, необходимо выполнить начальную настройку:

\textbf{Установка Ruby:} \url{https://www.ruby-lang.org/en/downloads/}

\textbf{Установка Git:} \url{https://git-scm.com/downloads}

\begin{minted}{bash}
git config --global user.name "Your Name"
git config --global user.email "your_email@whatever.com"
\end{minted}

\textbf{Настройка окончания строк}

Для Unix/Mac:
\begin{minted}{bash}
git config --global core.autocrlf input
git config --global core.safecrlf true
\end{minted}

Для Windows:
\begin{minted}{bash}
git config --global core.autocrlf true
git config --global core.safecrlf true
\end{minted}

\textbf{Установка Ruby}

Для выполнения заданий необходим установленный Ruby.

% \refFigure{screenshot_GI_1.png} представлен скриншот данного задания.
% \imageWithCaption{screenshot_GI_1.png}{Настройка имени, email и окончания строк. Ruby установлен.}
% Скриншот: вывод git config и ruby --version

\section{Lab 2: More Setup}
\textbf{Цель:} подготовить материалы для работы с учебным курсом.

\textbf{Загрузка и распаковка архива}

Скачайте архив: \url{https://gitimmersion.com/git_tutorial.zip}

Распакуйте его. Внутри будет папка \texttt{git\_tutorial} со следующими подкаталогами:
\begin{itemize}
  \item \texttt{html} — HTML-файлы курса. Откройте \texttt{html/index.html} в браузере.
  \item \texttt{work} — пустая рабочая папка. Здесь создаются репозитории.
  \item \texttt{repos} — готовые репозитории. Используйте их для восстановления прогресса.
\end{itemize}

% \refFigure{screenshot_GI_2.png} представлен скриншот данного задания.
% \imageWithCaption{screenshot_GI_2.png}{Структура распакованного архива git_tutorial.}
% Скриншот: дерево каталогов html, work и repos

\section{Lab 3: Create a Project}
\textbf{Цель:} создать репозиторий Git с нуля.

\textbf{Создание программы Hello, World}

Внутри рабочей директории создайте папку \texttt{hello}, затем в ней файл \texttt{hello.rb} со следующим содержимым:

\begin{minted}{bash}
mkdir hello
cd hello
# создать hello.rb со строкой:
puts "Hello, World"
\end{minted}

% \refFigure{screenshot_GI_3a.png} представлен скриншот данного задания.
% \imageWithCaption{screenshot_GI_3a.png}{Создана директория hello и файл hello.rb с кодом.}
% Скриншот: терминал и содержимое hello.rb

\textbf{Инициализация репозитория}

\begin{minted}{bash}
git init
\end{minted}

Git создаст скрытую папку \texttt{.git} и начнёт отслеживать изменения в проекте.

% \refFigure{screenshot_GI_3b.png} представлен скриншот данного задания.
% \imageWithCaption{screenshot_GI_3b.png}{Результат git init в директории hello.}
% Скриншот: вывод команды git init

\textbf{Добавление файла и первый коммит}

\begin{minted}{bash}
git add hello.rb
git commit -m "First Commit"
\end{minted}

После выполнения вы увидите сообщение о создании первого коммита.

% \refFigure{screenshot_GI_3c.png} представлен скриншот данного задания.
% \imageWithCaption{screenshot_GI_3c.png}{Добавление hello.rb и коммит.}
% Скриншот: команды git add и git commit с результатом

\section{Lab 4: Checking Status}
\textbf{Цель:} узнать, как проверять текущее состояние репозитория.

Выполните команду:
\begin{minted}{bash}
git status
\end{minted}

Она покажет, что изменений нет:
\begin{minted}{text}
On branch main
nothing to commit, working tree clean
\end{minted}

Это означает, что все изменения зафиксированы, рабочая директория чиста. Команду \texttt{git status} следует регулярно использовать для контроля актуального состояния проекта.

% \refFigure{screenshot_GI_4.png} представлен скриншот данного задания.
% \imageWithCaption{screenshot_GI_4.png}{Вывод git status после первого коммита.}

\section{Lab 5: Making Changes}
\textbf{Цель:} научиться отслеживать изменения в рабочей директории.

Измените файл \texttt{hello.rb}, чтобы он принимал аргумент из командной строки:

\begin{minted}{ruby}
puts "Hello, \#{ARGV.first}!"
\end{minted}

% \refFigure{screenshot_GI_5a.png} представлен скриншот данного задания.
% \imageWithCaption{screenshot_GI_5a.png}{Изменённый файл hello.rb с аргументом ARGV.}

Проверьте статус:
\begin{minted}{bash}
git status
\end{minted}

Git покажет, что файл изменён:
\begin{minted}{text}
Changes not staged for commit:
  (use "git add <file>..." to update what will be committed)
  (use "git restore <file>..." to discard changes in working directory)
	modified:   hello.rb
\end{minted}

Вывод подсказывает, как действовать дальше: либо зафиксировать изменения через \texttt{git add}, либо откатить их через \texttt{git restore}.

% \refFigure{screenshot_GI_5b.png} представлен скриншот данного задания.
% \imageWithCaption{screenshot_GI_5b.png}{Вывод git status после изменения hello.rb.}

\section{Lab 6: Staging Changes}
\textbf{Цель:} научиться подготавливать изменения для последующего коммита.

Добавьте изменения в индекс:
\begin{minted}{bash}
git add hello.rb
git status
\end{minted}

Результат:
\begin{minted}{text}
Changes to be committed:
  (use "git restore --staged <file>..." to unstage)
	modified:   hello.rb
\end{minted}

Теперь Git знает об изменении, но оно ещё не зафиксировано. Следующий коммит включит эти изменения. При необходимости можно отменить подготовку с помощью \texttt{git restore --staged}.

% \refFigure{screenshot_GI_6.png} представлен скриншот данного задания.
% \imageWithCaption{screenshot_GI_6.png}{Изменение hello.rb добавлено в индекс.}

\section{Lab 7: Staging and Committing}
\textbf{Цель:} понять преимущество раздельной подготовки и коммита изменений.

Допустим, вы изменили три файла: \texttt{a.rb}, \texttt{b.rb} и \texttt{c.rb}. Вы хотите закоммитить изменения в \texttt{a.rb} и \texttt{b.rb} вместе, а \texttt{c.rb} — отдельно.

Выполните:
\begin{minted}{bash}
git add a.rb
git add b.rb
git commit -m "Changes for a and b"
git add c.rb
git commit -m "Unrelated change to c"
\end{minted}

Благодаря отдельной стадии подготовки можно точно контролировать, какие изменения попадают в каждый коммит.

% \refFigure{screenshot_GI_7.png} представлен скриншот данного задания.
% \imageWithCaption{screenshot_GI_7.png}{Демонстрация коммитов для a.rb и b.rb отдельно от c.rb.}

\section{Lab 8: Committing Changes}
\textbf{Цель:} научиться выполнять коммиты в репозиторий.

Теперь зафиксируем подготовленные изменения, не используя флаг \texttt{-m}, чтобы открыть редактор по умолчанию:
\begin{minted}{bash}
git commit
\end{minted}

Откроется редактор с таким текстом:
\begin{minted}{text}
# Please enter the commit message for your changes. Lines starting
# with '#' will be ignored, and an empty message aborts the commit.
# On branch main
# Changes to be committed:
#   (use "git reset HEAD <file>..." to unstage)
#
#	modified:   hello.rb
\end{minted}

Введите комментарий: \texttt{Using ARGV}, затем сохраните и закройте редактор.

Результат:
\begin{minted}{text}
[main 569aa96] Using ARGV
 1 files changed, 1 insertions(+), 1 deletions(-)
\end{minted}

Проверьте статус:
\begin{minted}{bash}
git status
\end{minted}

\begin{minted}{text}
On branch main
nothing to commit, working tree clean
\end{minted}

Рабочая директория чиста.

% \refFigure{screenshot_GI_8a.png} представлен скриншот с открытым редактором.
% \imageWithCaption{screenshot_GI_8a.png}{Редактор с вводом комментария коммита.}

% \refFigure{screenshot_GI_8b.png} представлен скриншот вывода после сохранения.
% \imageWithCaption{screenshot_GI_8b.png}{Результат коммита и статус.}

\section{Lab 9: Changes, not Files}
\textbf{Цель:} понять, что Git работает с изменениями, а не с файлами.

Измените файл \texttt{hello.rb}, добавив значение по умолчанию:
\begin{minted}{ruby}
name = ARGV.first || "World"

puts "Hello, \#{name}!"
\end{minted}

Добавьте это изменение:
\begin{minted}{bash}
git add hello.rb
\end{minted}

Теперь добавьте комментарий к программе:
\begin{minted}{ruby}
# Default is "World"
name = ARGV.first || "World"

puts "Hello, \#{name}!"
\end{minted}

Проверьте статус:
\begin{minted}{bash}
git status
\end{minted}

\begin{minted}{text}
Changes to be committed:
	modified:   hello.rb

Changes not staged for commit:
	modified:   hello.rb
\end{minted}

% \refFigure{screenshot_GI_9a.png} представлен скриншот данного задания.
% \imageWithCaption{screenshot_GI_9a.png}{Файл hello.rb находится одновременно в staged и unstaged состояниях.}

Закоммитьте первое изменение (default value):
\begin{minted}{bash}
git commit -m "Added a default value"
git status
\end{minted}

\begin{minted}{text}
Changes not staged for commit:
	modified:   hello.rb
\end{minted}

Добавьте второе изменение:
\begin{minted}{bash}
git add .
git status
\end{minted}

\begin{minted}{text}
Changes to be committed:
	modified:   hello.rb
\end{minted}

Закоммитьте второе изменение:
\begin{minted}{bash}
git commit -m "Added a comment"
\end{minted}

% \refFigure{screenshot_GI_9b.png} представлен скриншот данного задания.
% \imageWithCaption{screenshot_GI_9b.png}{Пошаговая демонстрация индексации и фиксации изменений по отдельности.}

\section{Lab 10: History}
\textbf{Цель:} изучить, как просматривать историю изменений проекта.

Выполните команду:
\begin{minted}{bash}
git log
\end{minted}

Она выведет список всех коммитов, например:
\begin{minted}{text}
commit e4e3645... Added a comment
commit a6b268e... Added a default value
commit 174dfab... Using ARGV
commit f7c41d3... First Commit
\end{minted}

% \refFigure{screenshot_GI_10a.png} представлен скриншот истории коммитов через git log.
% \imageWithCaption{screenshot_GI_10a.png}{Полный вывод git log с комментариями.}

\textbf{Однострочная история}
\begin{minted}{bash}
git log --pretty=oneline
\end{minted}

\begin{minted}{text}
e4e3645 Added a comment
a6b268e Added a default value
174dfab Using ARGV
f7c41d3 First Commit
\end{minted}

% \refFigure{screenshot_GI_10b.png} представлен скриншот однострочной истории.
% \imageWithCaption{screenshot_GI_10b.png}{Вывод git log в однострочном формате.}

\textbf{Фильтрация и параметры}
Попробуйте следующие команды:
\begin{minted}{bash}
git log --pretty=oneline --max-count=2
git log --pretty=oneline --since='5 minutes ago'
git log --pretty=oneline --until='5 minutes ago'
git log --pretty=oneline --author=<ваше имя>
git log --pretty=oneline --all
\end{minted}

\textbf{Более продвинутый вывод}
\begin{minted}{bash}
git log --all --pretty=format:'%h %cd %s (%an)' --since='7 days ago'
\end{minted}

\textbf{Формат с графом и короткой датой}
\begin{minted}{bash}
git log --pretty=format:'%h %ad | %s%d [%an]' --graph --date=short
\end{minted}

\begin{minted}{text}
* e4e3645 2023-06-10 | Added a comment (HEAD -> main) [Jim Weirich]
* a6b268e 2023-06-10 | Added a default value [Jim Weirich]
* 174dfab 2023-06-10 | Using ARGV [Jim Weirich]
* f7c41d3 2023-06-10 | First Commit [Jim Weirich]
\end{minted}

% \refFigure{screenshot_GI_10c.png} представлен скриншот красивого графа коммитов.
% \imageWithCaption{screenshot_GI_10c.png}{Вывод git log с форматированием и графом.}

Для изучения всех возможных флагов используйте: \texttt{man git-log}

\textbf{Графические инструменты:} \texttt{gitk} и \texttt{gitx} (для macOS) позволяют удобно просматривать историю коммитов визуально.

\section{Lab 11: Aliases}
\textbf{Цель:} настроить псевдонимы (алиасы) для часто используемых команд Git.

\textbf{Создание алиасов в Git}
Добавьте в файл \texttt{.gitconfig} в домашней директории следующую секцию:

\begin{minted}{ini}
[alias]
  co = checkout
  ci = commit
  st = status
  br = branch
  hist = log --pretty=format:'%h %ad | %s%d [%an]' --graph --date=short
  type = cat-file -t
  dump = cat-file -p
\end{minted}

Теперь вы можете использовать сокращения:
\begin{itemize}
  \item \texttt{git co} вместо \texttt{git checkout}
  \item \texttt{git ci} вместо \texttt{git commit}
  \item \texttt{git st} вместо \texttt{git status}
  \item \texttt{git br} вместо \texttt{git branch}
  \item \texttt{git hist} — форматированный вывод истории коммитов
\end{itemize}

% \refFigure{screenshot_GI_11a.png} представлен скриншот содержания файла .gitconfig с добавленными алиасами.
% \imageWithCaption{screenshot_GI_11a.png}{Файл .gitconfig с определёнными алиасами.}

\textbf{Дополнительные алиасы для терминала (опционально)}
Для пользователей POSIX-подобных оболочек (например, bash/zsh) можно создать shell-алиасы, добавив следующее в \texttt{.profile} или \texttt{.bashrc}:

\begin{minted}{bash}
alias gs='git status '
alias ga='git add '
alias gb='git branch '
alias gc='git commit'
alias gd='git diff'
alias gco='git checkout '
alias gk='gitk --all&'
alias gx='gitx --all'

alias got='git '
alias get='git '
\end{minted}

Алиас \texttt{gco <ветка>} позволяет быстро переключаться между ветками.

% \refFigure{screenshot_GI_11b.png} представлен скриншот с shell-алиасами в .bashrc или .profile.
% \imageWithCaption{screenshot_GI_11b.png}{Shell-алиасы для Git команд.}

\textbf{Примечание:} в следующих заданиях будет использоваться алиас \texttt{hist} для просмотра истории. Убедитесь, что он настроен.

\section{Lab 12: Getting Old Versions}
\textbf{Цель:} научиться извлекать предыдущие версии файлов из репозитория.

\textbf{Посмотреть историю коммитов:}
\begin{minted}{bash}
git hist
\end{minted}

Вывод:
\begin{minted}{text}
* e4e3645 2023-06-10 | Added a comment (HEAD -> main) [Jim Weirich]
* a6b268e 2023-06-10 | Added a default value [Jim Weirich]
* 174dfab 2023-06-10 | Using ARGV [Jim Weirich]
* f7c41d3 2023-06-10 | First Commit [Jim Weirich]
\end{minted}

Найдите хеш первого коммита (последняя строка) и используйте его, чтобы вернуться к этой версии.

\textbf{Переход к старой версии:}
\begin{minted}{bash}
git checkout f7c41d3
cat hello.rb
\end{minted}

Вывод:
\begin{minted}{text}
puts "Hello, World"
\end{minted}

% \refFigure{screenshot_GI_12a.png} представлен скриншот команды checkout старого коммита.
% \imageWithCaption{screenshot_GI_12a.png}{Возврат к первой версии файла hello.rb.}

\textbf{Возврат к актуальной версии ветки main:}
\begin{minted}{bash}
git checkout main
cat hello.rb
\end{minted}

Вывод:
\begin{minted}{text}
# Default is "World"
name = ARGV.first || "World"

puts "Hello, #{name}!"
\end{minted}

% \refFigure{screenshot_GI_12b.png} представлен скриншот возврата на ветку main.
% \imageWithCaption{screenshot_GI_12b.png}{Содержимое hello.rb после возврата на main.}

\section{Lab 13: Tagging Versions}
\textbf{Цель:} научиться помечать коммиты для будущих ссылок.

Пометим текущую версию как v1:
\begin{minted}{bash}
git tag v1
\end{minted}

Теперь текущий коммит доступен по тегу \texttt{v1}.

\textbf{Тегирование предыдущих версий}

Воспользуемся символом \^ для ссылки на родителя текущей версии:
\begin{minted}{bash}
git checkout v1^
cat hello.rb
\end{minted}

% \refFigure{screenshot_GI_13a.png} представлен скриншот переключения на v1^.
% \imageWithCaption{screenshot_GI_13a.png}{Содержимое hello.rb в предыдущем коммите.}

Создадим тег \texttt{v1-beta}:
\begin{minted}{bash}
git tag v1-beta
\end{minted}

\textbf{Переключение между тегами}
\begin{minted}{bash}
git checkout v1
git checkout v1-beta
\end{minted}

% \refFigure{screenshot_GI_13b.png} представлен скриншот переключения между тегами.
% \imageWithCaption{screenshot_GI_13b.png}{Переход от v1 к v1-beta и обратно.}

\textbf{Просмотр тегов}
\begin{minted}{bash}
git tag
\end{minted}

\textbf{Просмотр тегов в истории}
\begin{minted}{bash}
git hist main --all
\end{minted}

% \refFigure{screenshot_GI_13c.png} представлен скриншот вывода истории с тегами.
% \imageWithCaption{screenshot_GI_13c.png}{История с тегами v1 и v1-beta.}

\section{Lab 14: Undoing Local Changes (before staging)}
\textbf{Цель:} научиться отменять изменения в рабочей директории до этапа индексации.

Переключитесь на ветку main:
\begin{minted}{bash}
git checkout main
\end{minted}

Измените файл \texttt{hello.rb}:
\begin{minted}{ruby}
# This is a bad comment.  We want to revert it.
name = ARGV.first || "World"

puts "Hello, \#{name}!"
\end{minted}

Проверьте статус:
\begin{minted}{bash}
git status
\end{minted}

\begin{minted}{text}
On branch main
Changes not staged for commit:
  (use "git add <file>..." to update what will be committed)
  (use "git restore <file>..." to discard changes in working directory)
	modified:   hello.rb
\end{minted}

Отмените изменения в файле:
\begin{minted}{bash}
git checkout hello.rb
git status
cat hello.rb
\end{minted}

\begin{minted}{text}
Updated 1 path from the index
On branch main
nothing to commit, working tree clean
# Default is "World"
name = ARGV.first || "World"

puts "Hello, \#{name}!"
\end{minted}

\section{Lab 15: Undoing Staged Changes (before committing)}
\textbf{Цель:} научиться отменять изменения, которые уже были проиндексированы (staged), но ещё не закоммичены.

\textbf{Измените файл hello.rb и проиндексируйте его:}

\begin{minted}{ruby}
# This is an unwanted but staged comment
name = ARGV.first || "World"

puts "Hello, \#{name}!"
\end{minted}

Проиндексируйте изменения:
\begin{minted}{bash}
git add hello.rb
\end{minted}

\textbf{Проверьте статус:}
\begin{minted}{bash}
git status
\end{minted}
\begin{minted}{text}
On branch main
Changes to be committed:
  (use "git restore --staged <file>..." to unstage)
	modified:   hello.rb
\end{minted}

% \refFigure{screenshot_GI_15a.png} представлен скриншот стадии после git add.
% \imageWithCaption{screenshot_GI_15a.png}{Проиндексированное изменение hello.rb.}

\textbf{Сброс проиндексированных изменений:}
\begin{minted}{bash}
git reset HEAD hello.rb
\end{minted}
\begin{minted}{text}
Unstaged changes after reset:
M	hello.rb
\end{minted}

% \refFigure{screenshot_GI_15b.png} представлен вывод git reset.
% \imageWithCaption{screenshot_GI_15b.png}{Сброс индексации файла hello.rb.}

\textbf{Отмена изменений в рабочем каталоге:}
\begin{minted}{bash}
git checkout hello.rb
git status
\end{minted}
\begin{minted}{text}
On branch main
nothing to commit, working tree clean
\end{minted}

% \refFigure{screenshot_GI_15c.png} показывает окончательное состояние рабочей директории.
% \imageWithCaption{screenshot_GI_15c.png}{Файл hello.rb восстановлен до коммита, рабочая директория чиста.}

\section{Lab 16: Undoing Committed Changes}
\textbf{Цель:} научиться отменять изменения, которые уже были зафиксированы в локальном репозитории.

\textbf{Сценарий:}
Иногда после коммита становится ясно, что изменения были ошибочными. В Git можно безопасно отменить такие коммиты путём создания нового коммита, отменяющего изменения предыдущего.

\textbf{1. Внесите изменение и закоммитьте его}
Измените файл \texttt{hello.rb} следующим образом:
\begin{minted}{ruby}
# This is an unwanted but committed change
name = ARGV.first || "World"

puts "Hello, #{name}!"
\end{minted}

\begin{minted}{bash}
git add hello.rb
git commit -m "Oops, we didn't want this commit"
\end{minted}

% \refFigure{screenshot_GI_16a.png} — внесение нежелательного коммита.
% \imageWithCaption{screenshot_GI_16a.png}{Файл изменён и зафиксирован с ошибочным комментарием.}

\textbf{2. Отмените коммит через revert}
Создайте новый коммит, отменяющий изменения предыдущего:
\begin{minted}{bash}
git revert HEAD
\end{minted}

Git откроет редактор — оставьте сообщение по умолчанию или измените его. Сохраните и выйдите.

Альтернатива:
\begin{minted}{bash}
git revert HEAD --no-edit
\end{minted}

% \refFigure{screenshot_GI_16b.png} — результат выполнения git revert.
% \imageWithCaption{screenshot_GI_16b.png}{Создан откатный коммит с сообщением Revert.}

\textbf{3. Проверка истории}
Проверьте историю коммитов:
\begin{minted}{bash}
git hist
\end{minted}

\begin{minted}{text}
* 8b71812 2023-06-10 | Revert "Oops, we didn't want this commit" (HEAD -> main) [Jim Weirich]
* 146fb71 2023-06-10 | Oops, we didn't want this commit [Jim Weirich]
* e4e3645 2023-06-10 | Added a comment (tag: v1) [Jim Weirich]
* a6b268e 2023-06-10 | Added a default value (tag: v1-beta) [Jim Weirich]
* 174dfab 2023-06-10 | Using ARGV [Jim Weirich]
* f7c41d3 2023-06-10 | First Commit [Jim Weirich]
\end{minted}

% \refFigure{screenshot_GI_16c.png} — история коммитов.
% \imageWithCaption{screenshot_GI_16c.png}{История: нежелательный и откатный коммиты присутствуют.}

\textbf{Вывод:} 
Операция \texttt{git revert} безопасна даже при работе с публичными ветками. Она не удаляет историю, а добавляет новый коммит, отменяющий предыдущий.

\section{Lab 17: Removing Commits from a Branch}
\textbf{\underline{Цель:}} удалить последние коммиты из ветки без сохранения их в истории.

\subsection*{Проверка истории коммитов}
Выполните:
\begin{minted}{bash}
git hist
\end{minted}

% \refFigure{screenshot_GI_17a.png} — история коммитов.
% \imageWithCaption{screenshot_GI_17a.png}{Вывод истории коммитов перед удалением.}

\subsection*{Создание тега на текущем коммите}
Отметим текущую вершину ветки тегом `oops`:
\begin{minted}{bash}
git tag oops
\end{minted}

\subsection*{Сброс ветки до нужного состояния}
Команда ниже удаляет последние коммиты и возвращает ветку к версии, отмеченной тегом `v1`:
\begin{minted}{bash}
git reset --hard v1
\end{minted}

Проверим историю снова:
\begin{minted}{bash}
git hist
\end{minted}

% \refFigure{screenshot_GI_17b.png} — история коммитов.
% \imageWithCaption{screenshot_GI_17b.png}{История после `reset --hard v1`.}

\subsection*{Проверка существования удалённых коммитов}
Удалённые коммиты всё ещё существуют в репозитории. Проверим:
\begin{minted}{bash}
git hist --all
\end{minted}

% \refFigure{screenshot_GI_17c.png} — история коммитов.
% \imageWithCaption{screenshot_GI_17c.png}{Вывод `git hist --all` показывает теги `v1` и `oops`.}

\subsection*{Важно:}
\begin{itemize}
  \item \textbf{git reset --hard} удаляет коммиты только из ветки, но не из репозитория.
  \item Коммиты без ссылок (тегов, веток) будут окончательно удалены после сборки мусора (garbage collection).
  \item Не рекомендуется использовать `reset` на публичных ветках, так как это нарушает историю для других пользователей.
\end{itemize}

\section{Lab 18: Remove the oops Tag}
\textbf{Цель:} удалить временный тег \texttt{oops}, чтобы он не сохранялся в истории репозитория.

\textbf{Удаление тега:}

Выполните команду удаления тега:
\begin{minted}{bash}
git tag -d oops
\end{minted}

\textbf{Проверка всех коммитов:}

\begin{minted}{bash}
git hist --all
\end{minted}

\textbf{Ожидаемый вывод:}
\begin{minted}{text}
$ git tag -d oops
Deleted tag 'oops' (was 8b71812)
$ git hist --all
* e4e3645 2023-06-10 | Added a comment (HEAD -> main, tag: v1) [Jim Weirich]
* a6b268e 2023-06-10 | Added a default value (tag: v1-beta) [Jim Weirich]
* 174dfab 2023-06-10 | Using ARGV [Jim Weirich]
* f7c41d3 2023-06-10 | First Commit [Jim Weirich]
\end{minted}

Теперь тег \texttt{oops} удалён и больше не отображается в истории.

% \refFigure{screenshot_GI_18.png} — история коммитов.
% \imageWithCaption{screenshot_GI_18.png}{Удаление тега oops и обновлённая история репозитория.}

\section{Lab 19: Amending Commits}

\textbf{Цель:} научиться изменять последний коммит.

\subsection*{Шаг 1: Добавление комментария автора и коммит}

Измените файл 	exttt{hello.rb}:
\begin{minted}{ruby}
# Default is World
# Author: Jim Weirich
name = ARGV.first || "World"

puts "Hello, #{name}!"
\end{minted}

Закоммитьте изменения:
\begin{minted}{bash}
git add hello.rb
git commit -m "Add an author comment"
\end{minted}

\subsection*{Шаг 2: Обнаружение ошибки}

Вы понимаете, что необходимо указать также email автора. Обновите файл:
\begin{minted}{ruby}
# Default is World
# Author: Jim Weirich (jim@somewhere.com)
name = ARGV.first || "World"

puts "Hello, #{name}!"
\end{minted}

\subsection*{Шаг 3: Изменение последнего коммита (amend)}

Сделайте amend последнего коммита:
\begin{minted}{bash}
git add hello.rb
git commit --amend -m "Add an author/email comment"
\end{minted}

\textbf{Ожидаемый вывод:}
\begin{minted}{text}
[main 186488e] Add an author/email comment
 Date: Sat Jun 10 03:49:14 2023 -0400
 1 file changed, 2 insertions(+), 1 deletion(-)
\end{minted}

\subsection*{Шаг 4: Проверка истории}
\begin{minted}{bash}
git hist
\end{minted}

\textbf{Ожидаемый вывод:}
\begin{minted}{text}
* 186488e 2023-06-10 | Add an author/email comment (HEAD -> main) [Jim Weirich]
* e4e3645 2023-06-10 | Added a comment (tag: v1) [Jim Weirich]
* a6b268e 2023-06-10 | Added a default value (tag: v1-beta) [Jim Weirich]
* 174dfab 2023-06-10 | Using ARGV [Jim Weirich]
* f7c41d3 2023-06-10 | First Commit [Jim Weirich]
\end{minted}

Теперь коммит с email заменил предыдущий, и в истории отображается только новая запись.

% Скриншоты:
% \refFigure{screenshot_GI_19a.png} — amend-коммит.
% \imageWithCaption{screenshot_GI_19a.png}{Изменённый коммит с email автора.}
% \refFigure{screenshot_GI_19b.png} — git hist.
% \imageWithCaption{screenshot_GI_19b.png}{История коммитов после amend.}

\section{Lab 20: Moving Files}

\textbf{Цель:} научиться перемещать файлы внутри репозитория.

\textbf{Перемещение hello.rb в каталог lib:}

Создайте новую директорию и выполните перемещение с помощью Git:
\begin{minted}{bash}
mkdir lib
git mv hello.rb lib
git status
\end{minted}

\textbf{Ожидаемый вывод:}
\begin{minted}{text}
$ mkdir lib
$ git mv hello.rb lib
$ git status
On branch main
Changes to be committed:
  (use "git restore --staged <file>..." to unstage)
	renamed:    hello.rb -> lib/hello.rb
\end{minted}

Команда \texttt{git mv} сообщает Git, что файл \texttt{hello.rb} был удалён и создан файл \texttt{lib/hello.rb}. Оба изменения сразу же попадают в индекс.

% \refFigure{screenshot_GI_20a.png} — статус после перемещения
% Скриншот статуса после перемещения
% \imageWithCaption{screenshot_GI_20a.png}{Вывод \texttt{git status} после перемещения файла в каталог lib.}

\textbf{Альтернативный способ перемещения:}

Можно воспользоваться обычными командами ОС и затем проиндексировать изменения вручную:
\begin{minted}{bash}
mkdir lib
mv hello.rb lib
git add lib/hello.rb
git rm hello.rb
\end{minted}

\textbf{Фиксация изменений:}
\begin{minted}{bash}
git commit -m "Moved hello.rb to lib"
\end{minted}

% \refFigure{screenshot_GI_20b.png} — фиксация перемещения
% \imageWithCaption{screenshot_GI_20b.png}{Фиксация перемещения файла в истории коммитов.}

\section{Lab 21: More Structure}

\textbf{Цель:} добавить новый файл в репозиторий.

\textbf{Установка Rake (при необходимости):}
\begin{minted}{bash}
gem install rake
\end{minted}

\textbf{Создание файла Rakefile:}
\begin{minted}{ruby}
#!/usr/bin/ruby -wKU

task :default => :run

task :run do
  require './lib/hello'
end
\end{minted}

\textbf{Добавление и коммит:}
\begin{minted}{bash}
git add Rakefile
git commit -m "Added a Rakefile."
\end{minted}

% \refFigure{screenshot_GI_21a.png} — коммит Rakefile.
% \imageWithCaption{screenshot_GI_21a.png}{Добавление Rakefile и фиксация изменений.}

\textbf{Запуск программы через Rake:}
\begin{minted}{bash}
rake
\end{minted}

\textbf{Ожидаемый вывод:}
\begin{minted}{text}
Hello, World!
\end{minted}

% \refFigure{screenshot_GI_21b.png} — запуск через rake.
% \imageWithCaption{screenshot_GI_21b.png}{Вывод `rake` показывает выполнение программы hello.}

\section{Lab 22: Git Internals — The .git Directory}

\textbf{Цель:} изучить структуру директории 	exttt{.git}.

\textbf{Просмотр содержимого директории .git:}
\begin{minted}{bash}
ls -C .git
\end{minted}
\begin{minted}{text}
$ ls -C .git
COMMIT_EDITMSG	config		index		objects
HEAD		description	info		packed-refs
ORIG_HEAD	hooks		logs		refs
\end{minted}

% \refFigure{screenshot_GI_22a.png} — структура .git директории.
% \imageWithCaption{screenshot_GI_22a.png}{Содержимое директории .git, хранящей внутренние данные репозитория.}

\textbf{Исследование object store:}
\begin{minted}{bash}
ls -C .git/objects
\end{minted}
\begin{minted}{text}
$ ls -C .git/objects
09	17	24	43	6b	97	af	c4	e7	pack
11	18	27	59	78	9c	b0	cd	f7
14	22	28	69	8b	a6	b5	e4	info
\end{minted}

% \refFigure{screenshot_GI_22b.png} — содержимое object store.
% \imageWithCaption{screenshot_GI_22b.png}{Папки с двухсимвольными именами — часть SHA1 хеша объектов.}

\textbf{Просмотр содержимого конкретной директории объектов:}
\begin{minted}{bash}
ls -C .git/objects/09
\end{minted}
\begin{minted}{text}
$ ls -C .git/objects/09
6b74c56bfc6b40e754fc0725b8c70b2038b91e
9fb6f9d3a104feb32fcac22354c4d0e8a182c1
\end{minted}

% \refFigure{screenshot_GI_22c.png} — SHA1-объекты внутри папки.
% \imageWithCaption{screenshot_GI_22c.png}{Файлы в поддиректории — оставшиеся 38 символов SHA1 хеша.}

\textbf{Просмотр конфигурационного файла:}
\begin{minted}{bash}
cat .git/config
\end{minted}
\begin{minted}{text}
$ cat .git/config
[core]
	repositoryformatversion = 0
	filemode = true
	bare = false
	logallrefupdates = true
	ignorecase = true
	precomposeunicode = true
[user]
	name = Jim Weirich
	email = jim (at) edgecase.com
\end{minted}

% \refFigure{screenshot_GI_22d.png} — содержимое .git/config.
% \imageWithCaption{screenshot_GI_22d.png}{Проектная конфигурация Git.}

\textbf{Просмотр ссылок на ветки и теги:}
\begin{minted}{bash}
ls .git/refs
ls .git/refs/heads
ls .git/refs/tags
cat .git/refs/tags/v1
\end{minted}
\begin{minted}{text}
$ ls .git/refs
heads
tags
$ ls .git/refs/heads
main
$ ls .git/refs/tags
v1
v1-beta
$ cat .git/refs/tags/v1
e4e3645637546103e72f0deb9abdd22dd256601e
\end{minted}

% \refFigure{screenshot_GI_22e.png} — refs и теги.
% \imageWithCaption{screenshot_GI_22e.png}{Каталоги ссылок на ветки и теги.}

\textbf{Файл HEAD:}
\begin{minted}{bash}
cat .git/HEAD
\end{minted}
\begin{minted}{text}
$ cat .git/HEAD
ref: refs/heads/main
\end{minted}

% \refFigure{screenshot_GI_22f.png} — файл HEAD.
% \imageWithCaption{screenshot_GI_22f.png}{HEAD указывает на активную ветку — main.}

\section{Lab 23: Git Internals — Working Directly with Git Objects}

\textbf{Цель:} исследовать структуру object store и научиться использовать SHA1-хэши для доступа к содержимому репозитория.

\textbf{Поиск последнего коммита:}
\begin{minted}{bash}
git hist --max-count=1
\end{minted}

\textbf{Ожидаемый вывод:}
\begin{minted}{text}
* cdceefa 2023-06-10 | Added a Rakefile. (HEAD -> main) [Jim Weirich]
\end{minted}

% \refFigure{screenshot_GI_23a.png} — последний коммит
% \imageWithCaption{screenshot_GI_23a.png}{Вывод `git hist --max-count=1` показывает последний коммит.}

\textbf{Просмотр commit-объекта:}
\begin{minted}{bash}
git cat-file -t cdceefa
git cat-file -p cdceefa
\end{minted}

% \refFigure{screenshot_GI_23b.png} — содержимое commit-объекта
% \imageWithCaption{screenshot_GI_23b.png}{Вывод `git cat-file -p` для последнего коммита.}

\textbf{Извлечение дерева:}
\begin{minted}{bash}
git cat-file -p 096b74c
\end{minted}

% \refFigure{screenshot_GI_23c.png} — дерево верхнего уровня
% \imageWithCaption{screenshot_GI_23c.png}{Дерево, содержащее `Rakefile` и каталог `lib`.}

\textbf{Извлечение содержимого lib:}
\begin{minted}{bash}
git cat-file -p e46f374
\end{minted}

% \refFigure{screenshot_GI_23d.png} — каталог lib
% \imageWithCaption{screenshot_GI_23d.png}{Каталог `lib` содержит файл `hello.rb`.}

\textbf{Просмотр содержимого hello.rb:}
\begin{minted}{bash}
git cat-file -p c45f26b
\end{minted}

% \refFigure{screenshot_GI_23e.png} — содержимое hello.rb
% \imageWithCaption{screenshot_GI_23e.png}{Содержимое файла `hello.rb`, извлечённое напрямую из Git-объекта.}

\textbf{Заключение:} мы вручную просмотрели commit-объект, связанное дерево, подкаталоги и файлы, используя только SHA1-хэши и команды `git cat-file`. Это демонстрирует, как устроен Git на внутреннем уровне.

\section{Lab 24: Creating a Branch}

\textbf{Цель:} научиться создавать локальные ветки в репозитории.

\textbf{Создание ветки:}
\begin{minted}{bash}
git checkout -b greet
git status
\end{minted}

% \refFigure{screenshot_GI_24a.png} — создание новой ветки.
% \imageWithCaption{screenshot_GI_24a.png}{Создана ветка `greet`, текущая ветка отображается в `git status`.}

\textbf{Добавление класса Greeter:}
\begin{minted}{ruby}
class Greeter
  def initialize(who)
    @who = who
  end
  def greet
    "Hello, #{@who}"
  end
end
\end{minted}

\begin{minted}{bash}
git add lib/greeter.rb
git commit -m "Added greeter class"
\end{minted}

% \refFigure{screenshot_GI_24b.png} — коммит с Greeter.
% \imageWithCaption{screenshot_GI_24b.png}{Создан файл `lib/greeter.rb`, закоммичен в ветке `greet`.}

\textbf{Изменение hello.rb для использования Greeter:}
\begin{minted}{ruby}
require 'greeter'

# Default is World
name = ARGV.first || "World"

greeter = Greeter.new(name)
puts greeter.greet
\end{minted}

\begin{minted}{bash}
git add lib/hello.rb
git commit -m "Hello uses Greeter"
\end{minted}

% \refFigure{screenshot_GI_24c.png} — обновление hello.rb.
% \imageWithCaption{screenshot_GI_24c.png}{Файл `lib/hello.rb` теперь использует класс Greeter.}

\textbf{Обновление Rakefile:}
\begin{minted}{ruby}
#!/usr/bin/ruby -wKU

task :default => :run

task :run do
  ruby '-Ilib', 'lib/hello.rb'
end
\end{minted}

\begin{minted}{bash}
git add Rakefile
git commit -m "Updated Rakefile"
\end{minted}

% \refFigure{screenshot_GI_24d.png} — обновлённый Rakefile.
% \imageWithCaption{screenshot_GI_24d.png}{Файл Rakefile обновлён для запуска через ruby-процесс с указанием каталога lib.}

\textbf{Итог:} ветка `greet` создана и содержит 3 новых коммита. Далее изучим переключение между ветками.

\section{Lab 25: Navigating Branches}

\textbf{Цель:} научиться переключаться между ветками в репозитории.

\textbf{Просмотр всех веток и коммитов:}
\begin{minted}{bash}
git hist --all
\end{minted}

% \refFigure{screenshot_GI_25a.png} — история всех веток.
% \imageWithCaption{screenshot_GI_25a.png}{Вывод `git hist --all` показывает две ветки: `main` и `greet`.}

\textbf{Переключение на ветку main:}
\begin{minted}{bash}
git checkout main
cat lib/hello.rb
\end{minted}

% \refFigure{screenshot_GI_25b.png} — содержимое hello.rb в ветке main.
% \imageWithCaption{screenshot_GI_25b.png}{Файл `lib/hello.rb` не использует класс `Greeter`, что подтверждает ветку `main`.}

\textbf{Переключение обратно на ветку greet:}
\begin{minted}{bash}
git checkout greet
cat lib/hello.rb
\end{minted}

% \refFigure{screenshot_GI_25c.png} — содержимое hello.rb в ветке greet.
% \imageWithCaption{screenshot_GI_25c.png}{Файл `lib/hello.rb` использует класс `Greeter`, что подтверждает ветку `greet`.}

\section{Lab 25: Navigating Branches}

\textbf{Цель:} научиться переключаться между ветками в репозитории.

\textbf{Просмотр всех веток и коммитов:}
\begin{minted}{bash}
git hist --all
\end{minted}

% \refFigure{screenshot_GI_25a.png} — история коммитов всех веток.
% \imageWithCaption{screenshot_GI_25a.png}{Вывод `git hist --all` показывает коммиты веток `main` и `greet`.}

\textbf{Переключение на ветку main:}
\begin{minted}{bash}
git checkout main
cat lib/hello.rb
\end{minted}

% \refFigure{screenshot_GI_25b.png} — содержимое hello.rb в ветке main.
% \imageWithCaption{screenshot_GI_25b.png}{Файл `lib/hello.rb` в ветке `main` не использует класс `Greeter`.}

\textbf{Возврат к ветке greet:}
\begin{minted}{bash}
git checkout greet
cat lib/hello.rb
\end{minted}

% \refFigure{screenshot_GI_25c.png} — содержимое hello.rb в ветке greet.
% \imageWithCaption{screenshot_GI_25c.png}{Файл `lib/hello.rb` в ветке `greet` использует класс `Greeter`.}

\section{Lab 26: Changes in Main}

\textbf{Цель:} научиться работать с несколькими ветками, содержащими различные (возможно конфликтующие) изменения.

\textbf{Переключение на ветку main:}
\begin{minted}{bash}
git checkout main
\end{minted}

% \refFigure{screenshot_GI_26a.png} — переключение на ветку main.
% \imageWithCaption{screenshot_GI_26a.png}{Команда `git checkout main` возвращает нас в основную ветку.}

\textbf{Создание файла README:}
\begin{minted}{text}
This is the Hello World example from the git tutorial.
\end{minted}

\textbf{Фиксация изменений:}
\begin{minted}{bash}
git add README
git commit -m "Added README"
\end{minted}

% \refFigure{screenshot_GI_26b.png} — коммит README.
% \imageWithCaption{screenshot_GI_26b.png}{Файл `README` добавлен и зафиксирован в ветке `main`.}

\section{Lab 27: Viewing Diverging Branches}

\textbf{Цель:} научиться просматривать ветки, расходящиеся в истории коммитов.

\textbf{Просмотр всех веток:}

Используем команду `git hist --all` для отображения истории всех веток и их расхождения.

\begin{minted}{bash}
git hist --all
\end{minted}

% \refFigure{screenshot_GI_27a.png} — история всех веток с отображением расхождений.
% \imageWithCaption{screenshot_GI_27a.png}{`git hist --all` с флагом `--graph` показывает расхождение между `main` и `greet`.}

\textbf{Объяснение:}

Флаг \texttt{--graph} добавляет визуальное отображение дерева коммитов в ASCII-графике, позволяя увидеть, где ветки расходятся. Флаг \texttt{--all} обеспечивает отображение всех веток, а не только текущей.

На изображении видно:

\begin{itemize}
  \item Ветка \texttt{main} содержит коммит \texttt{"Added README"}
  \item Ветка \texttt{greet} содержит 3 коммита, начиная с "Added greeter class"
  \item Общий предок — коммит \texttt{"Added a Rakefile"}
\end{itemize}

Теперь мы ясно видим, как развиваются параллельно две ветки в репозитории.

\section{Lab 28: Merging Branches}

\textbf{Цель:} объединить изменения из двух расходящихся веток в одну.

\textbf{Переход на ветку greet и слияние с main:}
\begin{minted}{bash}
git checkout greet
git merge main
git hist --all
\end{minted}

% \refFigure{screenshot_GI_28a.png} — успешное слияние ветки main в greet.
% \imageWithCaption{screenshot_GI_28a.png}{Вывод `git merge main` показывает создание merge-коммита и добавление файла README.}

% \refFigure{screenshot_GI_28b.png} — история коммитов после слияния.
% \imageWithCaption{screenshot_GI_28b.png}{`git hist --all` демонстрирует объединённую историю с merge-коммитом.}

\textbf{Комментарий:} регулярное слияние ветки main в рабочую ветку greet помогает отслеживать актуальные изменения в основной ветке и избегать конфликтов при финальной интеграции.

Позже мы рассмотрим альтернативу merge — \textit{rebase}, которая помогает сохранять историю более чистой.

% Lab 29: Creating a Conflict

\section{Lab 29: Creating a Conflict}

\textbf{Цель:} создать конфликт между ветками main и greet.

\textbf{Переключаемся на ветку main и вносим конфликтующие изменения:}
\begin{minted}{bash}
git checkout main
# Редактируем lib/hello.rb:
puts "What's your name"
my_name = gets.strip

puts "Hello, #{my_name}!"

git add lib/hello.rb
git commit -m "Made interactive"
\end{minted}

% \refFigure{screenshot_GI_29a.png} — изменения в hello.rb и коммит.
% \imageWithCaption{screenshot_GI_29a.png}{Интерактивная версия `hello.rb` коммичена в ветке `main`.}

\textbf{Просмотр истории всех веток:}
\begin{minted}{bash}
git hist --all
\end{minted}

% \refFigure{screenshot_GI_29b.png} — история коммитов.
% \imageWithCaption{screenshot_GI_29b.png}{История показывает коммит "Made interactive" в ветке `main`, не слитый в `greet`.}

\textbf{Заключение:} изменения в ветке `main` теперь конфликтуют с веткой `greet`, поскольку обе модифицируют `hello.rb`. Следующий шаг — разрешение конфликта.

% Lab 30: Resolving Conflicts

\section{Lab 30: Resolving Conflicts}

\textbf{Цель:} научиться разрешать конфликты при слиянии веток.

\textbf{Переход в ветку greet и попытка слияния с main:}
\begin{minted}{bash}
git checkout greet
git merge main
\end{minted}

% \refFigure{screenshot_GI_30a.png} — конфликт при слиянии.
% \imageWithCaption{screenshot_GI_30a.png}{Конфликт при слиянии: `lib/hello.rb` содержит несовместимые изменения.}

\textbf{Содержимое файла с конфликтом:}
\begin{minted}{ruby}
<<<<<<< HEAD
require 'greeter'

# Default is World
name = ARGV.first || "World"

greeter = Greeter.new(name)
puts greeter.greet
=======
# Default is World

puts "What's your name"
my_name = gets.strip

puts "Hello, #{my_name}!"
>>>>>>> main
\end{minted}

\textbf{Разрешаем конфликт: объединяем изменения вручную.}

\begin{minted}{ruby}
require 'greeter'

puts "What's your name"
my_name = gets.strip

greeter = Greeter.new(my_name)
puts greeter.greet
\end{minted}

\textbf{Фиксация разрешения конфликта:}
\begin{minted}{bash}
git add lib/hello.rb
git commit -m "Merged main fixed conflict."
\end{minted}

% \refFigure{screenshot_GI_30b.png} — успешная фиксация конфликта.
% \imageWithCaption{screenshot_GI_30b.png}{Коммит разрешения конфликта завершён: изменения из обеих веток объединены.}

\textbf{Заметка:} git позволяет использовать сторонние графические инструменты для слияния, см. официальную документацию: \\
\url{http://git-scm.com/book/en/v2/Customizing-Git-Git-Configuration#External-Merge-and-Diff-Tools}

% Lab 31: Rebasing vs Merging

\section{Lab 31: Rebasing vs Merging}

\textbf{Цель:} изучить различия между операциями \texttt{merge} и \texttt{rebase}.

\subsection*{Обсуждение}
Чтобы сравнить поведение \texttt{merge} и \texttt{rebase}, мы откатим состояние репозитория к моменту перед первым слиянием и повторим те же шаги, используя \texttt{rebase} вместо \texttt{merge}.

\textbf{Шаг 1: Откат веток к предыдущему состоянию}

\begin{minted}{bash}
git checkout greet
git reset --hard cab1837 # "Added greeter class"
git checkout main
git reset --hard 976950b # "Added README"

git checkout greet

git hist --all
\end{minted}

% \refFigure{screenshot_GI_31a.png} — история до ребейза.
% \imageWithCaption{screenshot_GI_31a.png}{История показывает две расходящиеся ветки до выполнения ребейза.}

\textbf{Шаг 2: Ребейз ветки greet на main}

\begin{minted}{bash}
git rebase main
\end{minted}

% \refFigure{screenshot_GI_31b.png} — успешный ребейз ветки greet.
% \imageWithCaption{screenshot_GI_31b.png}{Вывод `git rebase main` показывает переписанную историю коммитов.}

\textbf{Шаг 3: Просмотр истории после ребейза}

\begin{minted}{bash}
git hist --all
\end{minted}

% \refFigure{screenshot_GI_31c.png} — линейная история после ребейза.
% \imageWithCaption{screenshot_GI_31c.png}{История после `rebase` — линейная, без merge-коммита.}

\subsection*{Выводы}
\begin{itemize}
  \item \texttt{merge} сохраняет историю ветвления и создаёт merge-коммит.
  \item \texttt{rebase} переписывает историю коммитов, создавая более чистую и линейную историю.
  \item Для публичных веток предпочтительнее использовать \texttt{merge}, а для локальных — \texttt{rebase}.
\end{itemize}

% Lab 32: Resetting the Greet Branch

\section{Lab 32: Resetting the Greet Branch}

\textbf{Цель:} откатить ветку 	exttt{greet} к состоянию до первого слияния с веткой 	exttt{main}.

\textbf{Переходим в ветку greet и просматриваем историю:}
\begin{minted}{bash}
git checkout greet
git hist
\end{minted}

% \refFigure{screenshot_GI_32a.png} — история до reset.
% \imageWithCaption{screenshot_GI_32a.png}{История показывает коммиты до и после слияний, включая конфликт.}

\textbf{Откатываем ветку greet на коммит 	exttt{Updated Rakefile}:}
\begin{minted}{bash}
git reset --hard c1a7120
\end{minted}

% \refFigure{screenshot_GI_32b.png} — выполнение команды reset.
% \imageWithCaption{screenshot_GI_32b.png}{Ветка 	exttt{greet} сброшена на коммит 	exttt{Updated Rakefile}.}

\textbf{Проверка истории после отката:}
\begin{minted}{bash}
git hist --all
\end{minted}

% \refFigure{screenshot_GI_32c.png} — история после reset.
% \imageWithCaption{screenshot_GI_32c.png}{История коммитов ветки 	exttt{greet} до merge и конфликтов.}

\textbf{Вывод:} команда \texttt{git reset --hard} позволяет переместить указатель ветки на любой предыдущий коммит, полностью отменяя более поздние изменения в истории этой ветки.

% Lab 33: Resetting the Main Branch

\section{Lab 33: Resetting the Main Branch}

\textbf{Цель:} откатить ветку \texttt{main} до коммита, предшествующего конфликту.

\textbf{Переход на ветку main и просмотр истории:}
\begin{minted}{bash}
git checkout main
git hist
\end{minted}

% \refFigure{screenshot_GI_33a.png} — история коммитов ветки main.
% \imageWithCaption{screenshot_GI_33a.png}{Последний коммит — "Made interactive", предыдущий — "Added README".}

\textbf{Откат ветки main до коммита "Added README":}
\begin{minted}{bash}
git reset --hard 976950b
\end{minted}

% \refFigure{screenshot_GI_33b.png} — откат ветки main.
% \imageWithCaption{screenshot_GI_33b.png}{Ветка main успешно откатилась к коммиту "Added README".}

\textbf{Просмотр всей истории после отката:}
\begin{minted}{bash}
git hist --all
\end{minted}

% \refFigure{screenshot_GI_33c.png} — история всех веток после отката main.
% \imageWithCaption{screenshot_GI_33c.png}{main и greet теперь имеют общего предка и не содержат конфликтующих изменений.}

\textbf{Заключение:} теперь ветка \texttt{main} откатилась до безопасного состояния, и мы можем выполнять rebase без конфликтов.

\chapterConclusionSection{git_immersive_practice}

В четвёртой главе была реализована практическая часть, направленная на глубокое изучение внутренних механизмов Git — так называемой директории \texttt{.git}, структуры объектов, управления ветками, а также стратегий объединения и разрешения конфликтов.

Каждая лабораторная работа демонстрировала определённый аспект работы Git: от просмотра содержимого объектного хранилища и анализа коммитов до ручного разрешения конфликтов и сравнения подходов merge и rebase. Все задания были выполнены с фиксацией ключевых моментов с помощью скриншотов, сопровождаемых пояснительными подписями.

Практика показала, что понимание внутренних процессов Git даёт уверенность в управлении версиями, особенно в сложных ситуациях слияния или возврата к предыдущим состояниям проекта. Итогом главы стало формирование системного мышления в области контроля версий и готовность к более сложным DevOps-ориентированным задачам.



% -------------------------------------

\unnumberedChapter{Заключение и рекомендации}

\textbf{Общие выводы.}

В результате выполнения практической работы были достигнуты все поставленные цели. Студент овладел базовыми навыками работы с системой вёрстки \LaTeX{} и системой контроля версий Git \cite{latex_official_doc, progit_book}, а также закрепил знания с помощью платформы Learn Git Branching \cite{learngitbranching}.

На практике были выполнены следующие ключевые этапы:
\begin{itemize}
  \item подготовка отчётного документа в \LaTeX{} с использованием профессиональных пакетов для оформления;
  \item прохождение всех блоков интерактивной платформы Learn Git Branching;
  \item визуализация и пояснение каждой задачи с приложением скриншотов и кода;
  \item соблюдение требований к структуре, оформлению и библиографическому аппарату.
\end{itemize}

Полученные знания и навыки являются фундаментальными для последующей профессиональной деятельности в сфере IT и научных исследований. Документ может быть использован как образец оформления практических работ и отчётов.

\textbf{Рекомендации.}

\begin{itemize}
  \item Рекомендуется продолжить углублённое изучение Git, включая такие темы, как rebase, stash, cherry-pick и CI/CD-интеграции.
  \item Освоение более продвинутых возможностей \LaTeX{}, таких как TikZ, Beamer и автоматическая генерация диаграмм, позволит расширить инструментарий вёрстки.
  \item В рамках будущих курсов или проектов имеет смысл применить полученные навыки на реальных проектах, используя GitHub как платформу совместной работы.
\end{itemize}

\textbf{Дополнительные материалы.}

Весь исходный код работы, включая файлы \LaTeX{}, изображения и библиографию, доступен по следующей ссылке:
\begin{center}
\textbf{\github}
\end{center}

Скомпилированный итоговый документ в формате PDF можно скачать здесь:
\begin{center}
\textbf{\pdfLink}
\end{center}

\noindent Эти материалы предоставлены для свободного изучения, повторного использования и адаптации в образовательных целях.

% -------------------------------------

\bibliographyChapter % Библиография

% -------------------------------------

\end{document}
% vim: fdm=syntax
